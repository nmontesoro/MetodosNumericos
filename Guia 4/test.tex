\documentclass{article}
\usepackage[utf8]{inputenc}
\usepackage[a4paper, margin=2.5cm]{geometry}
\usepackage[spanish]{babel}
\usepackage{hyperref}
\usepackage{listings}
\usepackage[dvipsnames]{xcolor}

\hypersetup{
    colorlinks=true,
    linkcolor=blue,
    filecolor=blue,      
    urlcolor=blue,
}

\urlstyle{same}

\lstdefinestyle{mystyle}{
    backgroundcolor=\color{White},
    commentstyle=\color{OliveGreen},
    keywordstyle=\color{RubineRed},
    numberstyle=\tiny\color{Gray},
    stringstyle=\color{Orchid},
    basicstyle=\ttfamily\footnotesize,
    columns=flexible,
    breakatwhitespace=false,
    breaklines=true,
    captionpos=b,
    keepspaces=true,
    numbers=left,
    numbersep=5pt,
    showspaces=false,
    showstringspaces=false
    showtabs=false,
    tabsize=2
}

\lstset{style=mystyle}

\title{Primer parcial}
\author{Nazareno Montesoro - DNI 41.306.835}
\date{29 de abril de 2021}

\begin{document}
\maketitle

El código es este

\lstinputlisting[language=Matlab]{CalcRaizPosFalsa.m}

Y se puede llamar con

\begin{lstlisting}[language=Matlab]
    format compact
    % (a)
    H = hilb(5);
    [L, U, P] = lu(H)
    
    % (b)
    R = rand(5);
    [L, U, P] = lu(R)
    
    format
\end{lstlisting}

\section{Código utilizado}
El código utilizado en la resolución de estos ejercicios se encuentra en éstas páginas. 
También está disponible en Github, \url{https://github.com/nmontesoro/MetodosNumericos}.

\end{document}